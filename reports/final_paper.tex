%%%%%%%% DS 3001 Final Paper %%%%%%%%%%%%%%%%%

\documentclass{article}

% Recommended, but optional, packages for figures and better typesetting:
\usepackage{microtype}
\usepackage{graphicx}
\usepackage{subfigure}
\usepackage{booktabs} % for professional tables

\usepackage{longtable}
\usepackage{array}

% hyperref makes hyperlinks in the resulting PDF.
% If your build breaks (sometimes temporarily if a hyperlink spans a page)
% please comment out the following usepackage line and replace
% \usepackage{icml2025} with \usepackage[nohyperref]{icml2025} above.
\usepackage{hyperref}


% Attempt to make hyperref and algorithmic work together better:
\newcommand{\theHalgorithm}{\arabic{algorithm}}

% Use the following line for the initial blind version submitted for review:
% \usepackage{icml2025}

% If accepted, instead use the following line for the camera-ready submission:
\usepackage[accepted]{style/icml2025}

% For theorems and such
\usepackage{amsmath}
\usepackage{amssymb}
\usepackage{mathtools}
\usepackage{amsthm}

% if you use cleveref..
\usepackage[capitalize,noabbrev]{cleveref}

%%%%%%%%%%%%%%%%%%%%%%%%%%%%%%%%
% THEOREMS
%%%%%%%%%%%%%%%%%%%%%%%%%%%%%%%%
\theoremstyle{plain}
\newtheorem{theorem}{Theorem}[section]
\newtheorem{proposition}[theorem]{Proposition}
\newtheorem{lemma}[theorem]{Lemma}
\newtheorem{corollary}[theorem]{Corollary}
\theoremstyle{definition}
\newtheorem{definition}[theorem]{Definition}
\newtheorem{assumption}[theorem]{Assumption}
\theoremstyle{remark}
\newtheorem{remark}[theorem]{Remark}

% Todonotes is useful during development; simply uncomment the next line
%    and comment out the line below the next line to turn off comments
%\usepackage[disable,textsize=tiny]{todonotes}
\usepackage[textsize=tiny]{todonotes}


% The \icmltitle you define below is probably too long as a header.
% Therefore, a short form for the running title is supplied here:
\icmltitlerunning{Submission and Formatting Instructions for ICML 2025}

\begin{document}

\twocolumn[
\icmltitle{DS 3001 Final Project Paper}

% It is OKAY to include author information, even for blind
% submissions: the style file will automatically remove it for you
% unless you've provided the [accepted] option to the icml2025
% package.

% List of affiliations: The first argument should be a (short)
% identifier you will use later to specify author affiliations
% Academic affiliations should list Department, University, City, Region, Country
% Industry affiliations should list Company, City, Region, Country

% You can specify symbols, otherwise they are numbered in order.
% Ideally, you should not use this facility. Affiliations will be numbered
% in order of appearance and this is the preferred way.
\icmlsetsymbol{equal}{*}

\begin{icmlauthorlist}
\icmlauthor{Henry Allen}{equal,sch}
\icmlauthor{Kyra Lim}{equal,sch}
\icmlauthor{Emma Wunderly}{equal,sch}
\end{icmlauthorlist}

\icmlaffiliation{sch}{School of Data Science, University of Virginia, Charlottesville, VA, United States of America}

\icmlcorrespondingauthor{Henry Allen}{jpg7hy@virginia.edu}
\icmlcorrespondingauthor{Kyra Lim}{smy2qe@virginia.edu}
\icmlcorrespondingauthor{Emma Wunderly}{tyw3nq@virginia.edu}

% You may provide any keywords that you
% find helpful for describing your paper; these are used to populate
% the "keywords" metadata in the PDF but will not be shown in the document
\icmlkeywords{Machine Learning, ICML}

\vskip 0.3in
]

% this must go after the closing bracket ] following \twocolumn[ ...

% This command actually creates the footnote in the first column
% listing the affiliations and the copyright notice.
% The command takes one argument, which is text to display at the start of the footnote.
% The \icmlEqualContribution command is standard text for equal contribution.
% Remove it (just {}) if you do not need this facility.

%\printAffiliationsAndNotice{}  % leave blank if no need to mention equal contribution
\printAffiliationsAndNotice{\icmlEqualContribution} % otherwise use the standard text.

\begin{abstract}
Our Project explores refugee migration rates and how we can predict such rates based on origin country and asylum country freedom indexes. 
\end{abstract}

\section{Data}
\label{Data}
The data in our project consists of refugee migration statistics from United Nations High Commission for Refugees and Freedom Indexes from the Freedom House. With this Data we plan to investigate refugee rates from countries around the world, and eventually create a model that predicts migration rates based on the freedom indexes of ones home and host country.

The Freedom in the World dataset, published annually by Freedom House, provides scores and ratings from 2013 to 2021 for countries and territories. The data include measures such as political rights ratings, civil liberties ratings, and aggregate scores for subcategories like electoral process, political pluralism and participation, rule of law, and freedom of expression and belief. Each country-year entry also includes an overall status of Free, Partly Free, or Not Free. These variables allow us to capture the political and civil environment in both origin and asylum countries.

The UNHCR Persons of Concern dataset contains annual records beginning in 2004, organized by year, country of origin, and country of asylum. For each country pair, the dataset reports counts of various displaced groups, including refugees, asylum-seekers, internally displaced persons (IDPs), stateless persons, and others of concern. These values reflect the scale of displacement across borders and within states. These values provide the dependent variable in our analysis, which is the amount of refugee movement from one country to another in a given year.

The unit of analysis in our study is the country-year pair, which links refugee flows from an origin country to an asylum country in a given year with their corresponding freedom index values. This allows us to test hypotheses about how political repression, democratic governance, and civil liberties shape refugee migration. While the UNHCR data captures migration outcomes, the Freedom House data provides measures of the political conditions that may act as push and pull factors.

These datasets do consist limitations, such that refugee statistics often underrepresented the quantities of undocumented displacement, and the Freedom House scores, while widely used, are based partly on subjective assessments. Despite this, their examination provides a unique opportunity to connect quantitative migration patterns with qualitative assessments of political freedom.

\subsection{Data Cleaning}
The data cleaning process for the refugee and freedom index data involved several steps to ensure the data was ready for analysis. For the persons-of-concern.csv file, which contains the data regarding refugee migration, the initial loading and inspection revealed no immediate issues with missing values or data types, so no further cleaning was required. For the All-data-FIW-2013-2021.xlsx file, which contains the data regarding freedom indexes, initial attempts to load the data resulted in incorrect headers due to metadata at the beginning of the sheet. This was corrected by specifying the correct sheet name ('FIW13-21') and header row (row 0) during loading. The columns were then renamed for clarity, specifically 'C/T?' to 'Country Type', 'Country/Territory' to 'Country', and 'Total' to 'Freedom Total Score'. Finally, the first row, which contained descriptive information rather than data. We removed this so data handling could be smoother. Then relevant columns were selected and converted to appropriate numeric types. Finally we checked for null values and confirmed that there were no missing values in the selected columns after these steps.

\subsection{Data Dictionary}
This table provides a data dictionary for the Freedom in the World (FIW) dataset, defining the various columns and their meanings.

\begin{table}[H]

\caption{Data Dictionary}

\label{tab:data_dict}

\vskip 0.15in

\begin{center}

\begin{small}

\renewcommand{\arraystretch}{1.3} % more row spacing
\setlength{\tabcolsep}{6pt} % column padding

\begin{tabular}{|p{0.22\columnwidth}|p{0.68\columnwidth}|}
\hline
\textbf{Column Name} & \textbf{Description} \\
\hline

Year & The year of the data record. \\
\hline

Country of Asylum & The name of the country where a person is seeking or has received protection. \\
\hline

Country of Origin & The name of the country from which a person has been displaced. \\
\hline

Country of Asylum ISO & The ISO 3166-1 alpha-3 code for the country of asylum. \\
\hline

Country of Origin ISO & The ISO 3166-1 alpha-3 code for the country of origin. \\
\hline

Refugees & The total number of refugees. \\
\hline

Asylum-seekers & The total number of asylum-seekers. \\
\hline

IDPs & The total number of Internally Displaced Persons. \\
\hline

Other people in need of international protection & The total number of people in need of international protection who do not fall into the other categories. \\
\hline

Stateless persons & The total number of stateless persons. \\
\hline

Host community & The total number of individuals in the host community. \\
\hline

Others of concern & The total number of individuals who are not included in the main categories but are of concern to humanitarian organizations. \\
\hline

Country/Territory & The name of the country or territory. \\
\hline

Region & The geographic region of the country or territory. \\
\hline

C/T & Indicates if the entry is a country (c) or territory (t). \\
\hline

Edition & The year of the report. \\
\hline

Status & The overall freedom status: F (Free), PF (Partly Free), or NF (Not Free). \\
\hline

PR rating & The Political Rights Rating, on a scale of 1 (least free) to 7 (most free). \\
\hline

CL rating & The Civil Liberties Rating, on a scale of 1 (least free) to 7 (most free). \\
\hline

PR & The aggregate score for the Political Rights category, based on sub-scores from A, B, and C. \\
\hline

A1 to A3 & Sub-scores for the A. Electoral Process subcategory. \\
\hline

A & The aggregate score for the A. Electoral Process subcategory. \\
\hline

B1 to B4 & Sub-scores for the B. Political Pluralism and Participation subcategory. \\
\hline

B & The aggregate score for the B. Political Pluralism and Participation subcategory. \\
\hline

C1 to C3 & Sub-scores for the C. Functioning of Government subcategory. \\
\hline

C & The aggregate score for the C. Functioning of Government subcategory. \\
\hline

CL & The aggregate score for the Civil Liberties category, based on sub-scores from D, E, F, and G. \\
\hline

D1 to D4 & Sub-scores for the D. Freedom of Expression and Belief subcategory. \\
\hline

D & The aggregate score for the D. Freedom of Expression and Belief subcategory. \\
\hline

E1 to E3 & Sub-scores for the E. Associational and Organizational Rights subcategory. \\
\hline

E & The aggregate score for the E. Associational and Organizational Rights subcategory. \\
\hline

F1 to F4 & Sub-scores for the F. Rule of Law subcategory. \\
\hline

F & The aggregate score for the F. Rule of Law subcategory. \\
\hline

G1 to G4 & Sub-scores for the G. Personal Autonomy and Individual Rights subcategory. \\
\hline

G & The aggregate score for the G. Personal Autonomy and Individual Rights subcategory. \\
\hline

Add Q & The score for Additional Discretionary Question Q. (Formerly Add B) \\
\hline

Add A & The score for Additional Discretionary Question A. \\
\hline

Total & The aggregate score for all categories. \\
\hline

\end{tabular}

\end{small}

\end{center}

\vskip -0.1in

\end{table}


\section{Methods and Results}
\label{methods}


\section{Conclusion}
\label{conclusion}

% Acknowledgements should only appear in the accepted version.
% \section*{Acknowledgements}

% \textbf{Do not} include acknowledgements in the initial version of
% the paper submitted for blind review.

% If a paper is accepted, the final camera-ready version can (and
% usually should) include acknowledgements.  Such acknowledgements
% should be placed at the end of the section, in an unnumbered section
% that does not count towards the paper page limit. Typically, this will 
% include thanks to reviewers who gave useful comments, to colleagues 
% who contributed to the ideas, and to funding agencies and corporate 
% sponsors that provided financial support.

\section*{Impact Statement}
\label{Imapact Statement}

\nocite{*}

\bibliography{final_paper}
\bibliographystyle{style/icml2025}

%%%%%%%%%%%%%%%%%%%%%%%%%%%%%%%%%%%%%%%%%%%%%%%%%%%%%%%%%%%%%%%%%%%%%%%%%%%%%%%
%%%%%%%%%%%%%%%%%%%%%%%%%%%%%%%%%%%%%%%%%%%%%%%%%%%%%%%%%%%%%%%%%%%%%%%%%%%%%%%
% APPENDIX
%%%%%%%%%%%%%%%%%%%%%%%%%%%%%%%%%%%%%%%%%%%%%%%%%%%%%%%%%%%%%%%%%%%%%%%%%%%%%%%
%%%%%%%%%%%%%%%%%%%%%%%%%%%%%%%%%%%%%%%%%%%%%%%%%%%%%%%%%%%%%%%%%%%%%%%%%%%%%%%
\newpage
\appendix
\onecolumn
\section{Appendix}

You can have as much text here as you want. The main body must be at most $8$ pages long.
For the final version, one more page can be added.
If you want, you can use an appendix like this one.  

The $\mathtt{\backslash onecolumn}$ command above can be kept in place if you prefer a one-column appendix, or can be removed if you prefer a two-column appendix.  Apart from this possible change, the style (font size, spacing, margins, page numbering, etc.) should be kept the same as the main body.
%%%%%%%%%%%%%%%%%%%%%%%%%%%%%%%%%%%%%%%%%%%%%%%%%%%%%%%%%%%%%%%%%%%%%%%%%%%%%%%
%%%%%%%%%%%%%%%%%%%%%%%%%%%%%%%%%%%%%%%%%%%%%%%%%%%%%%%%%%%%%%%%%%%%%%%%%%%%%%%


\end{document}


% This document was modified from the file originally made available by
% Pat Langley and Andrea Danyluk for ICML-2K. This version was created
% by Iain Murray in 2018, and modified by Alexandre Bouchard in
% 2019 and 2021 and by Csaba Szepesvari, Gang Niu and Sivan Sabato in 2022.
% Modified again in 2023 and 2024 by Sivan Sabato and Jonathan Scarlett.
% Previous contributors include Dan Roy, Lise Getoor and Tobias
% Scheffer, which was slightly modified from the 2010 version by
% Thorsten Joachims & Johannes Fuernkranz, slightly modified from the
% 2009 version by Kiri Wagstaff and Sam Roweis's 2008 version, which is
% slightly modified from Prasad Tadepalli's 2007 version which is a
% lightly changed version of the previous year's version by Andrew
% Moore, which was in turn edited from those of Kristian Kersting and
% Codrina Lauth. Alex Smola contributed to the algorithmic style files.
